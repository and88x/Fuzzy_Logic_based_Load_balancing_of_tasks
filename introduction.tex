\section{Introduction}
\label{sec:intro}
\PARstart{T}{here} are many areas in which unmanned aerial vehicles are helpful to accomplish complex or risky tasks. 
Some of these fields are related to civil structures monitoring, mapping of urban and natural areas, search and rescue in emergency scenarios, and environmental monitoring. 
Also, a very important application of UAVs is related to source identification and location. 
A source could be radiation, acoustic signals, electromagnetic signals, or a chemical agent. 
Our work is related to this kind of applications, where the main objective is the location of an air pollutant source on an outdoor scenario considering realistic time constraints. 
The task is accomplished by a couple of autonomous quadcopters equipped with appropriate sensors which measure a specific pollutant. 
UAVs can move with 6 degrees of freedom and are capable to fly for limited time in different environments, making them suitable to take the pollutant samples.

In the literature, these  platforms are widely used.
For instance, \cite{li2020multi, yang2019towards, zou2015particle, nickels2019effective, chen2020combining, feng2019source} use multiple mobile robots to sample the field. 
In these works, global optimization algorithms command the movement of each robot. 
In \cite{li2020multi} details are given about a system incorporating a leading-follower behavior with a PSO algorithm, which is responsible of guiding the mobile robots. 
Each robot is considered as a particle of the swarm and the Schrödinger equation guides its movement.
The leader of the swarm is chosen depending on the global optimal position. 
The followers serve the leader providing measurements and navigating in the direction chosen by the leader. 
Authors of \cite{yang2019towards} use an indoor controlled environment to perform  source tracking. 
Two experiments are developed in that work: with and without airflow information. 
The airflow is varied with a displacement ventilation or mixing ventilation. 
For each experiment six terrestrial robots are guided in 3 phases: finding the plume (with a random divergence strategy), tracking the plume (by a standard and improved Whale Optimization Algorithm) and declaring the source.

In a platform with multiple agents, the cooperation is a powerful tool to succeed in the task of finding a source.
Research reports in \cite{facinelli2019cooperative, euler2017optimized, bayat2017environmental, li2017potential} focus on reducing their time to accomplish a mission and exchange information between each robotic agent. 
In \cite{facinelli2019cooperative} the agents share their position, velocity, and formation vector to perform a coordinated scanning of the search area. 
There are four UAV agents in this approach.
The exploration phase carries out three strategies: leader-follower, random walk scanning with feasible drone orientations, and Brownian motion behavior.
%determinates collision avoidance and boundary collision.
The exploration phase performs the following steps:
\begin{enumerate}
    \item The UAV that reacts to a gas measurement is transformed in a leader
    \item A circular formation around the leader is performed
    \item The swarm moves along a logarithmic spiral
    \item If the $i-th$ UAV detects a gas concentration, greater than the previous measures, that UAV is considered the new leader
\end{enumerate}

The plume is simulated with a Gaussian model, and the experiments assume that the gas concentration is a decreasing function of the distance from the source.

Previous strategies were proved on simulated environments or indoor controlled experiments.
However, in the literature is possible to find several works implemented in outdoor scenarios \cite{bayat2017environmental,villa2016development,ya2017uav,yungaicela2017design}. 
In these works, the focus is made on the construction of a platform with high maneuverability and capacity to sense air pollutant concentrations. 
With those platforms it is possible to execute the exploratory and exploitative strategies for source tracking. 
To carry out the experiments, it is necessary to consider all security aspects about the pollutant plume and the UAVs flight. 
Different pollutant sources have been considered in research works, such as sources of alcohol \cite{rossi2015autonomous}, sound \cite{hoshiba2017design}, or even it is considered to fly in zones where the presence of contamination is  known \cite{black2018adaption,yang2018natural,yang2017real}. 
%For a secure real flight, the local regulations need to be considered. For instance, some of the Mexican regulation are \cite{leyes_drones_1}: 
%\begin{itemize}
%\item The pilot must operate the aircraft within line of sight; must be able to see it throughout the flight.
%\item The pilot must maintain control of the flight path of the remotely piloted aircraft at all times.
%\item The aircraft must not be operated remotely, in open or closed places where there are more than 12 people.
%\end{itemize}

This paper presents results on the application of a novel intelligent strategy to locate an air pollutant source with two UAVs.
A simulation system based on two UAVs and the MAVLINK navigation protocol \cite{koubaa2019micro} was developed. 
The main purpose of this simulation is to allow a transparent migration of the developed scripts into a real platform, similar to that reported in \cite{yungaicela2017design}.
Additionally, in this work, an analysis of the performance efficiency of the UAVs to track and locate an air pollutant source is performed.
To this purpose, parameters like distance to the source, time to finish the exploration phase, highest contamination measure taken and ability to detect higher contaminant measurements will be statistically analyzed.   

Three different strategies for the exploration phase, each accompanied with its respective strategy for the exploitation phase, are tested in a scenario with a simulated pollutant plume. 
A punctual source and an advection-diffusion model generate this plume.
Here, wind data taken from real measurements is used, so the simulation behavior is closer to reality.

The main content of this paper is organized as follows: section 2 describes the simulation environment and the air pollutant distribution model, the construction of probabilistic map and the explanation of strategies. Section 3 presents the results obtained in experiments and the analysis. The conclusions and future work are shown in section 4.

%************************************************

