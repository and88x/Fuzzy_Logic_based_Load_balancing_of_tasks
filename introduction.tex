\section{Introduction}
\label{sec:intro}
\PARstart{T}{here} are many areas in which unmanned aerial vehicles are helpful to accomplish complex or risky tasks.
Some of these fields are related to civil structures monitoring, mapping of urban and natural areas, search and rescue in emergency scenarios, and environmental monitoring. 
Also, a very important application of UAVs is related to source identification and location. 
A source could be radiation, acoustic signals, electromagnetic signals, or a chemical agent. 
Our work is related to this kind of application, where the main objective is the location of an air pollutant source on an outdoor scenario considering realistic time constraints. 
The task is accomplished by a couple of autonomous quadcopters equipped with appropriate sensors which measure a specific pollutant.
UAVs can move with 6 degrees of freedom and are capable to fly for a limited time in different environments, making them suitable to take pollutant samples.

In the literature, these kinds of platforms are widely used. 
For instance, \cite{li2020multi, yang2019towards, zou2015particle, nickels2019effective, chen2020combining, feng2019source} use multiple mobile robots to sample the field.
In these works, global optimization algorithms command the movement of each robot. In \cite{li2020multi} details are given about a system incorporating a leading-follower behavior with a PSO algorithm responsible for guiding the mobile robots.
Each robot is considered as a particle of the swarm and the Schrödinger equation guides its movement. 
The leader of the swarm is chosen depending on the global optimal position.
The followers serve the leader providing measurements and navigating in the direction chosen by the leader. 
The authors of \cite{yang2019towards} use an indoor controlled environment to perform source tracking. 
Two experiments are developed in that work: with and without airflow information. 
The airflow is varied with displacement ventilation or mixing ventilation. 
For each experiment six terrestrial robots are guided in 3 phases: finding the plume (with a random divergence strategy), tracking the plume (by a standard and improved Whale Optimization Algorithm), and declaring the source.

In a platform with multiple agents, cooperation is a powerful tool to succeed in the task of finding a source.
Research reports in \cite{facinelli2019cooperative, euler2017optimized, bayat2017environmental, li2017potential} focus on reducing their time to accomplish a mission and exchange information between each robotic agent. 
In \cite{facinelli2019cooperative} the agents share their position, velocity, and formation vector to perform a coordinated scanning of the search area. 
There are four UAV agents in this approach. 
The exploration phase carries out three strategies: leader-follower, random walk scanning with feasible drone orientations, and Brownian motion behavior. 
The exploration phase performs the following steps:
\begin{enumerate}
    \item The UAV that reacts to a gas measurement is transformed into a leader
    \item A circular formation around the leader is performed
    \item The swarm moves along a logarithmic spiral
    \item If the ith UAV detects a gas concentration, greater than the previous measures, that UAV is considered the new leader 
\end{enumerate}

The plume is simulated with a Gaussian model, and the experiments assume that the gas concentration is a decreasing function of the distance from the source.

Previous strategies were proved in simulated environments or indoor controlled experiments. 
However, in the literature is possible to find several works implemented in
outdoor scenarios \cite{bayat2017environmental,villa2016development,ya2017uav,yungaicela2017design}. 
In these works, the focus is made on the construction of a platform with high maneuverability and capacity to sense air pollutant concentrations. 
With those platforms, it is possible to execute exploratory and exploitative strategies for source tracking.
Different pollutant sources have been considered in research works, such as sources of alcohol \cite{rossi2015autonomous}, sound \cite{hoshiba2017design}, or even it is considered to fly in zones where the presence of contamination is known \cite{black2018adaption,yang2018natural,yang2017real}.
It is common to find articles that use potential fields to implement the collision avoidance \cite{kristiansen2012operational,han2018small,fu2019pollution,marjovi2014optimal,budiyanto2015uav}.

To analyze the performance of each approach, some works consider advantageous conditions or assumptions like: 
\begin{itemize}
\item Take-offs from inside a plume \cite{vsmidl2013tracking,kristiansen2012operational} or relatively close to it \cite{euler2012cooperative}
\item The initial fly direction is towards the plume \cite{kristiansen2012operational,han2018small} 
\item The carried sensors have a high sensitivity \cite{lee2019dual}
\item The search area is not very large, so UAVs can detect the plume before flight time runs out \cite{fu2019pollution}
\end{itemize}

This paper presents results on the application of a novel intelligent strategy to locate an air pollutant source, without considering previous advantages.
This work considers:
\begin{itemize}
\item Take-offs in zones where there is not any trace of the pollutant plume
\item Path planning depends on clustering deterministic waypoints, but with random initial centroids (with k-means)
\item Parameters like maximum ground speed, the sensitivity of sensors, flying time, radio frequency coverage area, among others, are selected according to UAVs equal to the proposed by \cite{yungaicela2017design}
\item Search area is large enough (500m x 500m) for experiments to complete without detecting a contaminating plume 
\end{itemize}
Other feature of our work is that we use real measurements of wind magnitude to simulate the pollutant plume.
Also, we simulation environment use the MAVLINK navigation protocol \cite{koubaa2019micro}. 
The main objectives of these features are to show a wind behavior closer to reality and to allow a simple migration of the developed scripts to a real platform (like \cite{yungaicela2017design}).
Additionally, we performed an analysis of the performance efficiency of the UAVs to track and locate an air pollutant source. 
To this, parameters like distance to the source, time to finish the exploration phase, highest pollution measure taken, and ability to detect higher contaminant measurements will be statistically analyzed.

Three different strategies for the exploration phase, each accompanied with its respective strategy for the exploitation phase, are tested in a scenario with a simulated pollutant plume. 
A punctual source and an advection-diffusion model generate this plume. Here, wind data taken from real measurements is used, so the simulation behavior is closer to
reality.
To avoid collisions between both UAVs, they fly at different heights like in \cite{viseras2016decentralized}. 
These heights remain constant during flight time.
In half experiments height of one UAV matches with the pollutant source height.
No UAV matches with the source in the rest of the experiments.

The main content of this paper is organized as follows: section 2 describes the simulation environment and the air pollutant distribution model, the construction of a probabilistic map, and the explanation of strategies. 
Section 3 presents the results obtained in experiments and the analysis. 
The conclusions and future work are shown in section 4.
